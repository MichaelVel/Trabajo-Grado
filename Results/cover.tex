
\thispagestyle{empty}
	
	\begin{figure}[ht]
	   \centering
			\includegraphics[width=5cm]{unal.png}
			\label{EscudoUNAL}
	\end{figure}
	
	\begin{center}
	\vspace{0.5cm}
	\LARGE
	Selección de Rasgos Funcionales de las comunidades \\
	de macroinvertebrados para su uso como variables \\
	indicadoras de  calidad en dos ríos de la sabana \\
	de Bogotá
	
	\vspace{2.5cm}
	\Large
	\textbf{Michael Sebastián Velandia Garavito}

	\vspace{4.5cm}
	\normalsize
	Universidad Nacional de Colombia \\
    Facultad de Ciencias, Departamento de Biología\\
    Bogotá, Colombia
	

	\vspace{.5cm}
	\normalsize
	Septiembre de 2021
	
	\vspace{1cm}
	\end{center}

\newpage	
\thispagestyle{empty}
	
	\begin{center}
	\vspace{0.8cm}
	\LARGE
	Selección de Rasgos Funcionales de las comunidades \\
	de macroinvertebrados para su uso como variables \\
	indicadoras de  calidad en dos ríos de la sabana \\
	de Bogotá
	
	\vspace{0.8cm}
	\Large
	\textbf{Michael Sebastián Velandia Garavito}
	
	\vspace{2.7cm}	
    \normalsize
	Trabajo de investigación presentado como requisito parcial para \\
	optar al título de: 
	
	\vspace{0.2cm}
	\normalsize
	\textbf{Biólogo}
    
	\vspace{1cm}
	\normalsize
	Director:\\
    Ph.D. Gabriel Antonio Pinilla Agudelo

	\vspace{1.3cm}
	\normalsize
	Línea de Investigación:\\
    Indicadores Biológicos en Ecosistemas Acuáticos\\
    Grupo de Investigación:\\
    Biodiversidad, biotecnología y conservación de ecosistemas\\ 
    (Depto. Biología)


	\vspace{1.3cm}
	\normalsize
	Universidad Nacional de Colombia\\
    Ciencias, Departamento de Biología\\
    Bogotá, Colombia\\
    2021\\

	\vspace{1.5cm}
	\end{center}

\newpage
\thispagestyle{empty}
    
    \begin{center}
    
    \vspace{0.8cm}
	\Large
	\textbf{Agradecimientos}
	
	\vspace{0.8cm}
	\normalsize
    El autor agradece al Centro de Estudios en Hidrobiología de la Universidad Manuela 
    Beltrán (CEH-UMB) y en especial al profesor Ciromar Lemus, quien facilitó el acceso
    a las muestras de macroinvertebrados para su análisis y aportó la información de 
    las variables fisicoquímicas medidas in situ. También se agradece a la Universidad
    Nacional de Colombia, a la Facultad de Ciencias y al Departamento de Biología 
    por el apoyo logístico para el uso del estereoscopio empleado para la identificación 
    de los especímenes.
        
    \vspace{0.8cm}     
	\end{center}
	
\newpage
