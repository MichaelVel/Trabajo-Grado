% Options for packages loaded elsewhere
\PassOptionsToPackage{unicode}{hyperref}
\PassOptionsToPackage{hyphens}{url}
%
\documentclass[
]{article}
\usepackage{lmodern}
\usepackage{amssymb,amsmath}
\usepackage{ifxetex,ifluatex}
\ifnum 0\ifxetex 1\fi\ifluatex 1\fi=0 % if pdftex
  \usepackage[T1]{fontenc}
  \usepackage[utf8]{inputenc}
  \usepackage{textcomp} % provide euro and other symbols
\else % if luatex or xetex
  \usepackage{unicode-math}
  \defaultfontfeatures{Scale=MatchLowercase}
  \defaultfontfeatures[\rmfamily]{Ligatures=TeX,Scale=1}
\fi
% Use upquote if available, for straight quotes in verbatim environments
\IfFileExists{upquote.sty}{\usepackage{upquote}}{}
\IfFileExists{microtype.sty}{% use microtype if available
  \usepackage[]{microtype}
  \UseMicrotypeSet[protrusion]{basicmath} % disable protrusion for tt fonts
}{}
\makeatletter
\@ifundefined{KOMAClassName}{% if non-KOMA class
  \IfFileExists{parskip.sty}{%
    \usepackage{parskip}
  }{% else
    \setlength{\parindent}{0pt}
    \setlength{\parskip}{6pt plus 2pt minus 1pt}}
}{% if KOMA class
  \KOMAoptions{parskip=half}}
\makeatother
\usepackage{xcolor}
\IfFileExists{xurl.sty}{\usepackage{xurl}}{} % add URL line breaks if available
\IfFileExists{bookmark.sty}{\usepackage{bookmark}}{\usepackage{hyperref}}
\hypersetup{
  pdftitle={SELECCIÓN DE RASGOS FUNCIONALES DE LAS COMUNIDADES DE MACROINVERTEBRADOS PARA SU USO COMO VARIABLES INDICADORAS DE CALIDAD EN DOS RÍOS DE LA SABANA DE BOGOTÁ},
  pdfauthor={Michael Velandia},
  hidelinks,
  pdfcreator={LaTeX via pandoc}}
\urlstyle{same} % disable monospaced font for URLs
\usepackage[margin=1in]{geometry}
\usepackage{graphicx,grffile}
\makeatletter
\def\maxwidth{\ifdim\Gin@nat@width>\linewidth\linewidth\else\Gin@nat@width\fi}
\def\maxheight{\ifdim\Gin@nat@height>\textheight\textheight\else\Gin@nat@height\fi}
\makeatother
% Scale images if necessary, so that they will not overflow the page
% margins by default, and it is still possible to overwrite the defaults
% using explicit options in \includegraphics[width, height, ...]{}
\setkeys{Gin}{width=\maxwidth,height=\maxheight,keepaspectratio}
% Set default figure placement to htbp
\makeatletter
\def\fps@figure{htbp}
\makeatother
\setlength{\emergencystretch}{3em} % prevent overfull lines
\providecommand{\tightlist}{%
  \setlength{\itemsep}{0pt}\setlength{\parskip}{0pt}}
\setcounter{secnumdepth}{-\maxdimen} % remove section numbering

\title{SELECCIÓN DE RASGOS FUNCIONALES DE LAS COMUNIDADES DE MACROINVERTEBRADOS
PARA SU USO COMO VARIABLES INDICADORAS DE CALIDAD EN DOS RÍOS DE LA
SABANA DE BOGOTÁ}
\author{Michael Velandia}
\date{29/04/2021}

\begin{document}
\maketitle

\hypertarget{introducciuxf3n}{%
\subsubsection{Introducción}\label{introducciuxf3n}}

\% Macroinvertebrados como bioindicadores\\
Las condiciones ambientales de una gran cantidad de rios en los paises
tropicales se ha ido deteriorando a raiz del rapido crecimiento de las
poblaciones humanas, los cambios en el uso del suelo, el desarrollo
agricola e industrial, y las actividades extractivas (Dudgeon, 2017). Si
bien existen indices que hacen uso de parametros fisicos y quimicos para
la medición de la calidad de las aguas de los rios y su integridad, las
experiencias en paises templados (referencias) han demostrado que el uso
de los macroinvertebrados para el biomonitoreo presenta ventajas sobre
estos, pues integran información de cambios al corto y el largo plazo en
un amplio rango de variables ambientales (Dudgeon, 2017; Fierro et al.,
2017).

\%Limitaciones método tradicional El metodo tradicional usado para el
biomonitoreo con macroinvertebrados tiene un enfoque taxonomico basado
en la tolerancia que tienen ciertos taxones a la polución.

\%Rasgos funcionales como alternativa

El uso de macroinvertebrados para el biomonitoreo de la calidad de agua
ha estado tradicionalmente enmarcado en un enfoque taxonómico que se
basa en la tolerancia de ciertos taxones a la polución. Sin embargo, uno
de los mayores limitantes para esta aproximación es la dificultad de
hacer una determinación de los organismos de manera rápida y a un
suficiente nivel de detalle (género o especie) para usarlos en programas
de monitoreo(Cummins et al., 2015).

\%Perspectivas regionales y en el pais

Gran parte de los estudios que se han realizado sobre el uso de la
diversidad y los rasgos funcionales en bioindicación se han ejecutado en
países de zonas templadas (Charvet et al., 1998; Gayraud et al., 2003;
Statzner et al., 2005; Dolédec et al.,2006), siendo mucho menor la
cantidad de trabajos llevados a cabo en el trópico (Ding et al., 2017).
Una alta proporción de las investigaciones en el Neotrópico sobre este
tema se han hecho en Brasil, con algunos otros ejemplos en países como
Ecuador o Bolivia (Fossati et al., 2001; Cummins et al., 2005; Tomanova
et al., 2008). En Colombia se han realizado algunos acercamientos sobre
aspectos funcionales de los macroinvertebrados en la región insular
(Motta Díaz et al., 2020), en Antioquia (Toro et al., 2020) y en Boyacá
(Torres y Torres, 2016), pero en ninguno de ellos se ha buscado
identificar los rasgos más apropiados para la construcción de índices
biológicos.

Desde hace algunas décadas, el estudio de los rasgos funcionales de las
especies de sistemas lóticos ha demostrado una gran utilidad como un
buen indicador de la respuesta de las comunidades a diversos tipos de
disturbios, entre los que se encuentran los de naturaleza antrópica
(Ding et al., 2017), y en consecuencia se ha desarrollado un enfoque
funcional para el biomonitoreo de sistemas lóticos, principalmente en
zonas templadas, con algunos ejemplos en el Neotrópico (Tomanova et al.,
2008). Un primer paso para el desarrollo de un índice biológico de
invertebrados con un enfoque funcional de los ríos de la sabana de
Bogotá, es la evaluación de los rasgos funcionales de las comunidades de
dichos macroinvertebrados para conocer qué rasgos podrían responder
mejor a los cambios en los parámetros fisicoquímicos e hidrológicos de
los cursos de agua, por lo que es necesario empezar a abordar este tipo
de acercamientos en los ecosistemas acuáticos colombianos.

\% Definición de objetivos

\begin{comment}
 Objetivos: 
 -Evaluar la respuesta de los rasgos funcionales y de la diversidad funcional
  de las comunidades de macroinvertebrados a la variación en la calidad 
  del agua en dos ríos de la Sabana de Bogotá.

 -Establecer los rasgos funcionales de las comunidades de macroinvertebrados
  que reflejen mejor los cambios físicos, químicos e hidrológicos de los ríos
  Neusa y Frío.

 -Comparar la respuesta de los rasgos funcionales de los macroinvertebrados 
  ante los cambios en las variables fisicoquímicas del agua con la obtenida 
  por medio de otras variables tradicionalmente medidas en dichos indicadores
  biológicos.
\end{comment}

\hypertarget{metodologuxeda}{%
\subsubsection{Metodología}\label{metodologuxeda}}

\end{document}
